\documentclass[11pt,a4paper]{article}
%% -*- latex-command: xelatex -*-
\usepackage{xunicode}
\usepackage{xltxtra}
\usepackage{polyglossia}
\setmainlanguage{english}
\newfontfamily\cyrillicfont{Arial}
\setotherlanguages{russian,russian}
\usepackage[absolute]{textpos}
\usepackage{setspace}
\usepackage{color}
\usepackage{array}
\usepackage{graphicx}
\usepackage{multicol}
\usepackage[xetex,unicode=true,hyperfootnotes=false,colorlinks=true,citecolor=black,filecolor=black,linkcolor=black,urlcolor=black,pdfborder={0 0 0}]{hyperref}
\hypersetup{%
pdfstartview={Fit},%
pdftitle={Стратегии фокализации в предложениях частного вопроса (на примере некоторых языков Северного Кавказа и манден)},%
pdfauthor={Nina Sumbatova,, Valentin Vydrin},%
pdfkeywords={constituent question, interrogation, focus, Caucasian languages, Maninka, Malinke, Mande},%
pdfcreator={HAL},%
pdfproducer={PDFLaTeX},%
pdfsubject={Humanities and Social Sciences/Linguistics}}
\urlstyle{same}
\pagestyle{empty}
\setlength{\topmargin}{-2cm}
\setlength{\headheight}{0cm}
\setlength{\headsep}{0cm}
\setlength{\textheight}{18.7cm}
\setlength{\textwidth}{17cm}
\setlength{\oddsidemargin}{0cm}
\setlength{\evensidemargin}{0cm}
\setlength{\parindent}{0.25in}
\setlength{\parskip}{0.25in}
\newlength{\posXlogo}
\newlength{\posYlogo}
\setlength{\posXlogo}{2cm}
\setlength{\posYlogo}{23.5cm}
\newlength{\posXident}
\newlength{\posYident}
\setlength{\posXident}{2cm}
\setlength{\posYident}{19cm}
\newlength{\posXhal}
\newlength{\posYhal}
\setlength{\posXhal}{2cm}
\setlength{\posYhal}{23.5cm}
\newlength{\posXcc}
\newlength{\posYcc}
\setlength{\posXcc}{2cm}
\setlength{\posYcc}{\dimexpr\paperheight-4cm\relax}
%\usepackage{pdfpages}
\begin{document}
\topskip0pt\vspace*{0cm}
\begin{center}
\href{https://hal.archives-ouvertes.fr}{\includegraphics[width=0.25\textwidth]{hal-ao.jpg}} \\
\vfill
\begin{doublespace}
{\LARGE \textbf{\begin{russian}Стратегии фокализации в предложениях частного вопроса (на примере некоторых языков Северного Кавказа и манден)\end{russian}}} \\
{\Large Nina Sumbatova,, Valentin Vydrin}
\end{doublespace}
\end{center}
\includegraphics[width=8pt]{triangle.png}{\Large \textbf{~To cite this version:}} \\\\
\begin{tabular}{|p{\textwidth}}
{Nina Sumbatova,, Valentin Vydrin. \begin{russian}Стратегии фокализации в предложениях частного вопроса (на примере некоторых языков Северного Кавказа и манден) . Языки Дальнего Востока, Юго-Восточной Азии и Западной Африки. XII Международная конференция, Nov 2016, Moscou, Russia. Институт стран Азии и Африки МГУ, Восточный ф-т СПбГУ, Языки Дальнего Востока, Юго-Восточной Азии и Западной Африки. Материалы XII Международной конференции (Москва, 16--17 ноября 016 года). pp.238-246, 2016, Языки Дальнего Востока, Юго-Восточной Азии и Западной Африки. Материалы XII Международной конференции (Москва, 16--17 ноября 016 года). \textless{}http://www.iaas.msu.ru/index.php/ru/anonsy/1439-lesewa-12\textgreater{}. \textless{}halshs-01481128\textgreater{}\end{russian}} \\
\end{tabular}
\vfill
\begin{textblock*}{\textwidth}(\posXident , \posYident)
\begin{center}
\begin{doublespace}
{\Large \textbf{HAL Id: halshs-01481128}} \\
{\Large \textbf{\url{https://halshs.halpreprod.archives-ouvertes.fr/halshs-01481128}}} \\
Submitted on 2 Mar 2017
\end{doublespace}
\end{center}
\end{textblock*}
\begin{textblock*}{\textwidth}(\posXhal , \posYhal)
\begin{multicols}{2}
\textbf{HAL} is a multi-disciplinary open access archive for the deposit and dissemination of scientific research documents, whether they are published or not. The documents may come from teaching and research institutions in France or abroad, or from public or private research centers.
\vfill
\columnbreak
L'archive ouverte pluridisciplinaire \textbf{HAL}, est destinée au dépôt et à la diffusion de documents scientifiques de niveau recherche, publiés ou non, émanant des établissements d'enseignement et de recherche français ou étrangers, des laboratoires publics ou privés.
\end{multicols}
\end{textblock*}
\end{document}
