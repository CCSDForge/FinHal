\documentclass{article}
\usepackage{fontspec}
\defaultfontfeatures{Mapping=tex-text}

\usepackage[margin=4cm]{geometry}
\usepackage{amsthm}

\title{Οἱ πρῶτοι ἀριθμοί}
\date{20 Μαρτίου 2008}

\begin{document}

\maketitle

\textbf{Θεώρημα 1.} «Οἱ πρῶτοι ἀριθμοὶ πλείους εἰσὶ παντὸς τοῦ προτεθέντος
πλήθους πρώτων ἀριθμῶν.» (\emph{Στοιχεία}, Εὐκλείδης). Ἄρα, τὸ σύνολο τῶν
πρώτων ἀριθμῶν εἶναι ἄπειρο.

\textsc{Ἀπόδειξη. } Ἔστω $P$ ἕvα πεπερασμένο σύνολο $\{p_1, \dots, p_k\}$
πρώτων ἀριθμῶν. Θεωροῦμε τὸv ἀκέραιον ἀριθμόν $N := p_1\cdots p_k + 1$, ποὺ
εἶναι μεγαλύτερος τοῦ $1$. Τότε ὑπάρχει ἕνα πρῶτο διαιρέτη $q$ τοῦ $N$.
Ἄv $q = p_i$, τότε $q | N - p_1\cdots p_k$, ἄρα $q | 1$, καὶ αὑτὸ εἶναι
ἄτοπο. $\qed$

\end{document}
