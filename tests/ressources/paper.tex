\documentclass[11pt,a4paper]{article}
\pdfoutput=1
\usepackage[blocks]{authblk}
\usepackage[utf8]{inputenc}
\usepackage{fancyhdr}
\usepackage{amsfonts,amsmath,amssymb}
\usepackage[pdftex,%
            unicode=true,%
            hyperfootnotes=false,%
            colorlinks=true,%
		    citecolor=black,%
		    filecolor=black,%
		    linkcolor=black,%
		    urlcolor=black,%
		    pdfborder={0 0 0}]{hyperref}
\hypersetup{%
pdfstartview={Fit},%
pdftitle={L'Investissement des migrants marocains dans le pays d'origine ou l'affirmation '' de soi chez soi ''},%
pdfauthor={Hicham Jamid},%
pdfkeywords={},%
pdfcreator={Sciencesconf.org},%
pdfproducer={PDFLaTeX},%
pdfsubject={Ce que font les diasporas}
}
\urlstyle{same}
\pagestyle{empty}

\setlength{\topmargin}{0cm}
\setlength{\headheight}{0cm}
\setlength{\headsep}{0cm}
\setlength{\textheight}{24.7cm}
\setlength{\textwidth}{16cm}
\setlength{\oddsidemargin}{0cm}
\setlength{\evensidemargin}{0cm}
\setlength{\parindent}{0.25in}
\setlength{\parskip}{0.25in}

\lhead{}
\chead{}
\rhead{}
\lfoot{}
\cfoot{}
\rfoot{sciencesconf.org:forumdiaspora:145631}

\renewcommand{\abstractname}{Résumé}

\begin{document}

\title{\textbf{L'Investissement des migrants marocains dans le pays d'origine ou l'affirmation '' de soi chez soi ''}}

\date{}

%% authors
\author[1]{Hicham Jamid\footnote{\protect\label{speaker}Intervenant}}
	%% affiliations
	\affil[1]{{\small Laboratoire Interdisciplinaire pour la Sociologie Economique (LISE) -- CNRS : UMR3320, Conservatoire National des Arts et Métiers [CNAM] -- Conservatoire National des Arts et Métiers 2 rue Conté - 1LAB40 75003 Paris, France}}

\maketitle
\thispagestyle{fancy}


\begin{abstract}
Apr{\`e}s des ann{\'e}es pass{\'e}es {\`a} l'{\'e}tranger, certains migrants marocains d{\'e}cident de regagner, de mani{\`e}re d{\'e}finitive ou temporaire, le Royaume[1]. Aujourd'hui, ils sont de plus en plus nombreux {\`a} se mobiliser sous forme de r{\'e}seaux, d'associations ou m{\^e}me {\`a} titre individuel, afin de contribuer au d{\'e}veloppement {\'e}conomique et social du pays, particuli{\`e}rement de celui de leurs r{\'e}gions d'origine ; et cela gr{\^a}ce, entre autres, {\`a} leurs transferts d'argent et leurs investissements. \\ Souvent, c'est l'immobilier qui constitue le secteur {\'e}conomique phare de l'emploi des {\'e}conomies des migrants marocains (Hamdouch et El Ftouh, 2009). Cela est incontestablement d{\^u} {\`a} la dimension symbolique de ce secteur. Pour ces migrants, construire une grande maison dans leur village d'origine ou/et dans un centre urbain est le moyen le plus s{\^u}r pour afficher la r{\'e}ussite de leur exp{\'e}rience migratoire, mat{\'e}rialiser {\'e}conomiquement leur succ{\`e}s, mais aussi symboliquement leur fort attachement {\`a} leur territoire d'origine [Villanova et \textit{al, }1994 ; Charef, 1999, 2003 ; Mekki, 2012]. Cette obsession d'investir dans le foncier peut s'expliquer {\'e}galement par les profils socioprofessionnels de ces migrants, qui sont, dans la plupart des cas, des travailleurs non qualifi{\'e}s, enrichis dans leurs pays d'immigration et qui ont du mal {\`a} se convertir en hommes d'affaires au Maroc. Toutefois, au c{\oe}ur des dynamiques actuelles de ses migrations internationales, le Maroc constitue une destination attractive pour un nouveau profil d'entrepreneurs migrants qui portent de plus en plus d'int{\'e}r{\^e}t aux secteurs {\'e}conomiques prolifiques. Dot{\'e}s d'une inclination plus entrepreneuriale que leurs a{\^i}n{\'e}s, certains entrepreneurs migrants se dirigent davantage vers le secteur tertiaire. \\ \\ Dans le secteur tertiaire, le tourisme (Chattou, 2011), le commerce et les services, constituent les activit{\'e}s {\'e}conomiques qui suscitent le plus grand int{\'e}r{\^e}t chez les migrants qui se lancent dans l'aventure de l'entrepreneuriat au Maroc. Il s'agit des projets touristiques (h{\^o}tels, auberges, maisons d'h{\^o}tes...etc.), des commerces de d{\'e}tail et de proximit{\'e}, des caf{\'e}s, des boulangeries, des restaurants, des agences de voyages...etc. ; souvent g{\'e}r{\'e}s par des membres de la famille rest{\'e}s sur place ou pour {\^e}tre autog{\'e}r{\'e}s apr{\`e}s un retour au pays par les migrants eux-m{\^e}mes. Pour eux, le choix de ce domaine {\'e}conomique est judicieux puisqu'il ne sollicite ni un capital initial {\'e}lev{\'e} ni des comp{\'e}tences et des ressources humaines soutenues. \\ \\ Ainsi, c'est cette nouvelle figure d'entrepreneurs migrants que ma communication se propose de traiter, et cela {\`a} travers le parcours d'une cinquantaine d'investisseurs enqu{\^e}t{\'e}s g{\'e}rant diff{\'e}rents projets commerciaux dans la ville d'Agadir. La probl{\'e}matique de cette pr{\'e}sentation se concentre autour de trois questions fondamentales : qui sont ces entrepreneurs migrants ? Pour quelles raisons ont-ils fait le choix d'investir dans le secteur du commerce ? Quelles sont les formes, les caract{\'e}ristiques et le mode de fonctionnement {\'e}conomique de leurs appareils commerciaux ? \\ \\  \\ [1] Malheureusement, les r{\'e}sultats du dernier Recensement g{\'e}n{\'e}ral de la population et de l'habitat au Maroc, r{\'e}alis{\'e} en septembre 2014, n'ont pas encore affich{\'e} les donn{\'e}es concernant les Marocains de retour. Selon le recensement de 2004, l'effectif de migrants de retour a atteint 165.416 personnes, dont 10 \% ont choisi la r{\'e}gion d'Agadir pour s'y installer. \\  
\end{abstract}

\end{document}
